\anonsection{Введение}


В связи с большим количеством программного обеспечения, разрабатываемого на языках высокого уровня становится вопрос об эффективности используемых алгоритмов и их оптимальной настройке.

Множество алгоритмов, в том числе машинного обучения, могут требовать различные предположения, веса, характеристики для их работы. Параметры алгоритмов, значения которых определяются непосредственно перед запуском данных алгоритмов, называются гиперпараметрами. 

Обычно слово "гиперпараметр" рассматривается в контексте машинного обучения, в данной же работе данное понятие будет рассмотрено для более общего контекста. В простейшем случае гиперпараметрами могут быть аргументы алгоритмов для вычисления математических функций. Например, для алгоритма вычисляющего значения функции $f(x, y) = \sin(x) + \sin(y)$ гиперпараметрами будут вещественнозначные аргументы $x$ и $y$.

 Гиперпараметры в том числе используются компиляторами (например GCC/Clang) для настройки процесса компиляции и линковки (оптимизационные флаги). В машинном обучении это могут быть веса нейронов в нейронных сетях и ошибка на выходе сети, расстояние между объектами в кластеризации и т.д. В генетических алгоритмах это может быть количество особей в популяции, настройки мутаций и т.д. 
 
 Целью данной работы является разработка программной платформы для оптимизации гиперпараметров. 
 
 Для достижения поставленное цели необходимо решить следующие задачи: 
 
 \begin{itemize}
 	\item Выполнить постановку задачи;
 	\item Изучить основные подходы для решения данной задачи;
 	\item Рассмотреть существующие решения;
 	\item Выполнить проектирование программной платформы, а также произвести её разработку. 
 \end{itemize}
 
 \textcolor{blue}{Описать разделы ВКР}


\clearpage